\section{Introduction} \label{sec:intro}

The resistive switching phenomenon of metal oxide materials has been studied for a half-century. The negative resistance have been already observed in some metal-oxide-metal(MIM) structures at early 1960's. Based on the hysteretic resistance switching behaviors, researchers firstly proposed to use the MIM structure for memory applications in 1967~\cite{first1,first2}. However, it was not until the early 21st century that the practical ReRAM application has been fabricated~\cite{fab1,fab2}. The oxide materials evolve from complex perovskite oxides to more easily fabricated binary TMO at early 2000's. The advantages of TMOs, such as good compatibility to CMOS processes and thermal/chemical resilience, make TMOs the most competitive materials for MIM structure ReRAM cell. Many TMOs was tested in Baek's work~\cite{fab2} and the fully CMOS technology compatible, NiO based ReRAM was demonstrated with operation voltage below 3V and switching current below 2mA. After that, a bunch of TMO based ReRAM with various materials, such as CuOx, WOx, HfOx, TiOx, and TaOx, have been demonstrated~\cite{CuOx,WOx,HfOx,TiTa,TaOx,TaOx433}. These TMO based ReRAMs have shown excellent features, including low power, fast access speed, small cell size, good scalability, as well as back-end-of-the-line (BEOL) CMOS process compatibility. For example, Panasonic published a paper at the International Solid-State Circuit Conference (ISSCC) this year that presented a multi-layer ReRAM macro with 443MB/s throughput at 8.2ns pulse width~\cite{TaOx433}. XXXXXXXXXXXXXXXXXXXXXX considered as a ......

Traditional memory technologies, such as SRAM, DRAM, and FLASH, suffers from both soft errors and hard errors. The soft error is a random, recoverable upsetting of the information store in memory cell, while the hard error is a permanent corruption of the memory cell results from physical defect. Although the emerging non-volatile memory technologies are not charge-based storage, they also suffer from the soft error and hard error results from the physical characteristics of the cell. The presence of hard error normally results from the limited endurance compared to DRAM and SRAM technologies. However, the cause of soft error is distinctive for each NVM. For example, the soft error of Phase-Change Memory (PCM) comes from the resistance shift behaviors of the amorphous phase of chalcogenide materials, as well as the thermal disturbance from adjacent cells. For Spin-Transfer Torque RAM (STT-RAM), the stochastic properties implies that both of the write and read operation can bring in soft error. Similar to other NVM technologies, the ReRAM also suffers from the soft errors and hard errors originated from different physical mechanisms. Specifically, the soft errors of ReRAM cell result from the retention failures of the cell. And the hard error always comes from the limited endurance of the cell. In the presence of both the soft errors and hard errors, the reliability of ReRAM array, especially for the most area/cost efficient cross-point structure ReRAM array, becomes a serious design challenge. In this work, we systematically summarized the mechanism of both soft and hard errors of ReRAM cell and proposed a unified model to characterize the failure behaviors of different types of hard errors. By using the model, the impact of soft and hard errors on the cross-point ReRAM array is detailed studied.



The remainder of the paper is organized as follows. In Section~\ref{sec:prel}, the background of the metal oxide ReRAM cell and cross-point architecture is introduced. xxxxxx. The experimental results and discussion are provided in Section~\ref{sec:expe}. We conclude our work with Section~\ref{sec:conc}. 