\subsection{Memory Structure of ReRAM Array}

Almost all of the random access memories are organized as a matrix-like structure: one memory cell along with its access device (normally a MOSFET) are located in each intersection of horizontal wordlines and vertical bitlines. Similarly, the ReRAM array is also has this NOR-type structure. There are two potential structures of a ReRAM array: the MOSFET-accessed structure (1T1R structure) and the cross-point structure. In the MOSFET-accessed structure, each ReRAM cell has a dedicated MOSFET as its access device. The advantage of this structure is that it is very easy to control each cell in the array independently without crosstalk results from the sneak current in cross-point structure. However, in the MOSFET-accessed structure, the size of the MOSFET should be designed large enough to satisfy the current requirement of the SET/RESET operation. Therefore, the total area of the ReRAM array often determined by the access devices instead of the ReRAM cells, which seriously harmed the area advantage of the ReRAM devices. On the other hand, the cross-point structure is a more area efficient structure compared to MOSFET-accessed structure. As shown in Fig.~\ref{fig:overview}(b), in the cross-point structure, each ReRAM cell is sandwiched by TE and BE at each cross-point of the array without access device. In this structure, each cell only occupy am area of $4F^2$ (F is the feature size of the fabrication technology), which is the theoretical smallest cell area for a single layer single level memory array. For example, Hynix and HP Labs has already demonstrated an 2Mb $4F^2$ cross-point ReRAM chip based on 54nm technology~\cite{TiTa}. In addition, the good 3-D stackability can further reduce the effective cell area. An 64MB $0.5F^2$ cross-point CMOx ReRAM chip was demonstrated by Unity Semiconductor at 2010~\cite{Unity}.


As mentioned, the write operations (SET and RESET) of a ReRAM cell require external voltage across the cell with specified magnitude and duration. To write a cell in the cross-point array, the wordline and bitline where the cell located should be activated at specified write voltage. In addition, all of the unselected wordlines and bitlines are set to a certain voltage or left floating to guarantee that all of the unselected cells are not disturbed. There are several write schemes of the cross-point array. One of the most common scheme for bipolar cross-point array is called HWHB scheme: during the write operation, the selected wordline and bitline are activated at write voltage $V_{write}$ or $0$, while all of the unselected wordlines and bitlines are half biased at $V_{write}/2$. In this scheme, the write voltage $V_{write}$ or -$V_{write}$ is fully applied across the selected cell. The other cells located at the same wordline and bitline with the selected cell are half biased at $V_{write}/2$. And there are no voltage drop across all of the other cells. However, even with proper write schemes, the sneak current at the half selected cells are significant. For a 512x512 array, the sneak current is more than 500 times larger than the SET/RESET current, bring in huge area overhead of the voltage drivers and unnecessary energy consumption at the half selected cells. Therefore, in order to implement a practical cross-point array with acceptable overhead, a large nonlinearity of the ReRAM cell is essential. The resistance of a ReRAM cell with nonlinearity increases with the reducing of applied voltage. In this case, the sneak current at half selected cells will be reduced significantly. The read operation is to bias the selected wordline at $V_{Read}$ and ground all of the other wordlines and bitlines. Then the state of the selected cell is read out by sense amplifier connected to the selected bitline.

%The nonlinearity coefficient is proposed to

Although the cross-point structure suffers from aforementioned
shortcomings, its area efficiency is an attractive superiority compared to other emerging NVMs. In addition, the BEOL CMOS process capability makes it possible to implement the array on top of (part of) the peripheral circuity, further improving the area efficiency. Since the chip area is directly related to the cost-per-bit metric, the cost advantage make the cross-point ReRAM a promising candidate for DRAM or FLASH replacement.

